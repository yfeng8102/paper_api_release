%% 
%% Copyright 2019-2020 Elsevier Ltd
%% 
%% This file is part of the 'CAS Bundle'.
%% --------------------------------------
%% 
%% It may be distributed under the conditions of the LaTeX Project Public
%% License, either version 1.2 of this license or (at your option) any
%% later version.  The latest version of this license is in
%%    http://www.latex-project.org/lppl.txt
%% and version 1.2 or later is part of all distributions of LaTeX
%% version 1999/12/01 or later.
%% 
%% The list of all files belonging to the 'CAS Bundle' is
%% given in the file `manifest.txt'.
%% 
%% Template article for cas-sc documentclass for 
%% double column output.

%\documentclass[a4paper,fleqn,longmktitle]{cas-sc}
\documentclass[a4paper,fleqn]{cas-sc}

\usepackage[numbers]{natbib}
%\usepackage[authoryear]{natbib}
%\usepackage[authoryear,longnamesfirst]{natbib}

% my package
\usepackage{lineno}
\usepackage{setspace}

%%%Author definitions
%\def\tsc#1{\csdef{#1}{\textsc{\lowercase{#1}}\xspace}}
%\tsc{WGM}
%\tsc{QE}
%\tsc{EP}
%\tsc{PMS}
%\tsc{BEC}
%\tsc{DE}
%%%

% Uncomment and use as if needed
%\newtheorem{theorem}{Theorem}
%\newtheorem{lemma}[theorem]{Lemma}
%\newdefinition{rmk}{Remark}
%\newproof{pf}{Proof}
%\newproof{pot}{Proof of Theorem \ref{thm}}

\begin{document}

% my command
\linenumbers

\let\WriteBookmarks\relax
\def\floatpagepagefraction{1}
\def\textpagefraction{.001}

% Short title
\shorttitle{API release of Pickering emulsions}

% Short author
\shortauthors{Y. Feng et~al.}

% Main title of the paper
\title [mode = title]{Monte Carlo simulation of Pickering particle dissolution and API release in Pickering Emulsions}                      
% Title footnote mark
% eg: \tnotemark[1]
%\tnotemark[1,2]

% Title footnote 1.
% eg: \tnotetext[1]{Title footnote text}
% \tnotetext[<tnote number>]{<tnote text>} 
%\tnotetext[1]{This document is the results of the research project funded by the National Science Foundation.}

%\tnotetext[2]{The second title footnote which is a longer text matter to fill through the whole text width and overflow into another line in the footnotes area of the first page.}


% First author
%
% Options: Use if required
% eg: \author[1,3]{Author Name}[type=editor,
%       style=chinese,
%       auid=000,
%       bioid=1,
%       prefix=Sir,
%       orcid=0000-0000-0000-0000,
%       facebook=<facebook id>,
%       twitter=<twitter id>,
%       linkedin=<linkedin id>,
%       gplus=<gplus id>]

% First author
\author[]{Yi Feng}[orcid=0000-0001-9359-567X]
\ead{yi.feng@polito.it}
\credit{Conceptualization of this study, Methodology, Software}

% Second author
\author[]{Antonio Buffo}[orcid=0000-0002-4152-0593]
\ead{antonio.buffo@polito.it}
\credit{Data curation, Writing - Original draft preparation}

\author[]{Elena Simone}[orcid=0000-0003-4000-2222]
\ead{elena.simone@polito.it}
\credit{Data curation, Writing - Original draft preparation}
\cormark[1]

% Address/affiliation
\affiliation[]{organization={Department of Applied Science and Technology, Politecnico di Torino},
    addressline={Corso Duca degli Abruzzi, 24}, 
    city={Torino},
    % citysep={}, % Uncomment if no comma needed between city and postcode
    postcode={10129}, 
    % state={},
    country={Italy}}

% Corresponding author text
\cortext[cor1]{Corresponding author}

% Here goes the abstract
\begin{abstract}
This template helps you to create a properly formatted \LaTeX\ manuscript.

\noindent\texttt{\textbackslash begin{abstract}} \dots 
\texttt{\textbackslash end{abstract}} and
\verb+\begin{keyword}+ \verb+...+ \verb+\end{keyword}+ 
which
contain the abstract and keywords respectively. 

\noindent Each keyword shall be separated by a \verb+\sep+ command.
\end{abstract}

% Use if graphical abstract is present
% \begin{graphicalabstract}
% \includegraphics{figs/grabs.pdf}
% \end{graphicalabstract}

% Research highlights
\begin{highlights}
\item Random sequential adsorption gives coverages of Pickering emulsions
\item API release from a population of Pickering emulsions is studied
\item Pickering particle dissolution induced coverage changes are included
\end{highlights}

% Keywords
% Each keyword is seperated by \sep
\begin{keywords}
Pickering emulsions \sep API release \sep particle dissolution \sep Monte Carlo method
\end{keywords}


\maketitle

\doublespacing

\section{Introduction}

\section{Methodology}

\subsection{Geometric modeling of Pickering emulsion}

\subsection{Dissolution of Pickering particles}

The dissolution of a particle can be described by the following mass transfer equation \cite{bird2002_}:
\begin{equation}
    \frac{\mathrm{d} m_\mathrm{p}}{\mathrm{d} t} = - k_\mathrm{p} A_\mathrm{p} \left(C_\mathrm{sat} - C_{\mathrm{p}, \mathrm{c}} \right),
    \label{eq: particle mass transfer equation}
\end{equation}
where $m_\mathrm{p}$ is the particle mass, $k_\mathrm{p}$ is the particle mass transfer coefficient, $A_\mathrm{p}$ is the particle surface area, $C_{\mathrm{sat}}$ is the saturation concentration, and $C_{\mathrm{p}, \mathrm{c}}$ is the concentration in the continuous (bulk) phase. 

For a capsule-shaped particle consisting of two semi-spherical ends and a cylindrical section, it is natural to assume that the particle surface shrinks along its normal direction during dissolution. Therefore, the height of the cylindrical section, $l_\mathrm{c}$, remains constant during dissolution because the top and bottom faces of the cylindrical section are covered by the semi-spherical ends. It can be proved that the particle surface, of both the semi-spherical and cylindrical parts, shrinks at the same rate as (see Appendix \ref{app: particle dissolution}):
\begin{equation}
    \frac{\mathrm{d} d_\mathrm{p} }{\mathrm{d} t} = -\frac{2 k_\mathrm{p} \left(C_\mathrm{sat} - C_{\mathrm{p}, \mathrm{c}} \right) }{\rho_\mathrm{p}},
    \label{eq: particle diameter change rate}
\end{equation}
where $\rho_\mathrm{p}$ is the particle density. The dissolution process can be governed by the reaction mechanism or the diffusion mechanism \cite{mullin2001_}. For reaction-controlled dissolution, it is assumed to be a first-order process, and a constant particle mass transfer coefficient $k_\mathrm{p}$ will be used. For a diffusion-controlled dissolution, the particle mass transfer coefficient can be written as:
\begin{equation}
    k_\mathrm{p} = \frac{\mathrm{Sh} \mathcal{D}_{\mathrm{p}, \mathrm{c}}}{d_{32, \mathrm{p}}},
\end{equation}
where $\mathrm{Sh}=2$ is the Sherwood number, $\mathcal{D}_{\mathrm{p}, \mathrm{c}}$ is the particle diffusivity in the continuous phase, and $d_{32, \mathrm{p}}=6V_\mathrm{p}/A_\mathrm{p}$ is the Sauter mean diameter of the particle with $V_\mathrm{p}=\pi d_\mathrm{p}^3/6 + \pi d_\mathrm{p}^2l_\mathrm{c}/4$ be the particle volume and $A_\mathrm{p}=\pi d_\mathrm{p}^2 + \pi d_\mathrm{p}l_\mathrm{c}$ be the particle surface area. The reaction-controlled dissolution process is size-independent, while the diffusion-controlled dissolution process is size-dependent. In experiments, the reacting $k_\mathrm{p}$ is generally easier to measure by monitoring the particle concentration in the continuous phase. The diffusing $k_\mathrm{p}$ is relatively difficult to measure because it requires accurate measurement of particle diffusivity and size distribution in addition to particle concentration. Based on this consideration, the reaction mechanism will be adopted in most cases, whereas the diffusion mechanism will be used only for comparison with the reaction mechanism.

Furthermore, Pickering particles are adsorbed at the Pickering emulsion droplet surfaces. Only a porportion of the particle surface is exposed into the continuous phase, while the rest is inside the disperse droplet. In this work, particles are assumed to be only dissoluble in the continuous phase. The proportion exposed to the continuous phase affects the dissolution rate. This proportion depends on the particle contact angle. In this work, the proportion of the available surface area for dissolution is calculated using the following equation
\begin{equation}
    k_\mathrm{a} = \frac{\pi d_\mathrm{p}^2 \frac{1 - \cos\theta}{2} + \theta d_\mathrm{p} l_\mathrm{c}}{\pi d_\mathrm{p}^2 + \pi d_\mathrm{p}l_\mathrm{c}}
    \label{eq: ka}
\end{equation}
where $\theta$ is the contact angle of particle. Eq.~\eqref{eq: ka} depends only on particle sizes and contact angle, which implies a planar interface.

Finally, the diameter and mass change rate of the $j$-th Pickering particle adsorbed on the $i$-th Pickering emulsion droplet surface are written as:
\begin{equation}
    \frac{\mathrm{d} d_{\mathrm{p}, i, j} }{\mathrm{d} t} = -\frac{2 k_{\mathrm{p}, i, j} k_{\mathrm{a}, i, j} \left(C_\mathrm{sat} - C_{\mathrm{p}, \mathrm{c}} \right) }{\rho_\mathrm{p}},
    \label{eq: ddpdt}
\end{equation}
\begin{equation}
    \frac{\mathrm{d} m_{\mathrm{p}, i, j} }{\mathrm{d} t} = \frac{\pi}{2} \rho_\mathrm{p} \left(  d_{\mathrm{p}, i, j}^2 + d_{\mathrm{p}, i, j} l_{\mathrm{c}, i, j} \right)\frac{\mathrm{d} d_{\mathrm{p}, i, j} }{\mathrm{d} t}.
\end{equation}

The particle concentration in the continuous phase can be evaluated using the particle mass conservation as:
\begin{equation}
    \frac{\mathrm{d} C_{\mathrm{p}, \mathrm{c}}}{\mathrm{d} t} = -\frac{1}{V_\mathrm{c}} \sum_{i=1}^{N_\mathrm{d}} \sum_{j=1}^{N_{\mathrm{p}, i}} \frac{\mathrm{d} m_{\mathrm{p}, i, j}}{\mathrm{d} t},
\end{equation}
where $C_{\mathrm{p}, \mathrm{c}}$ is the particle concentration in the continuous phase and $V_\mathrm{c}$ is the volume of the continuous phase, $N_\mathrm{d}$ is the droplet number and $N_{\mathrm{p}, i}$ is the number of particles adsorbed on the $i$-th droplet.

Particle dissolution affects API release from the Pickering emulsion droplets through coverage. The coverage of the $i$-th droplet can be calculated as:
\begin{equation}
    \phi_i = \sum_{j=1}^{N_{\mathrm{p}, i}} \varphi(r_{\mathrm{p}, i , j}, l_{\mathrm{c}, i,j}, \theta, r_{\mathrm{d}, i}),
\end{equation}
where $\varphi(r_{\mathrm{p}, i , j}, l_{\mathrm{c}, i,j}, \theta, r_{\mathrm{d}, i})$ is the coverage of the $j$-th particle adsorbed on the $i$-th droplet, depengind on particle sizes, contact angle, and droplet size. It is calculated using numerical ingeraval, whose details are provided in \cite{feng2025_Phys.Rev.E}.

\subsection{Release of API from Pickering emulsions}

The release of API from the $i$-th dispersed droplet to the continuous phase can be described using the interface transport concentration equation \cite{bird2002_} as:
\begin{equation}
    \frac{\mathrm{d} C_{\mathrm{API}, \mathrm{d},i}}{\mathrm{d} t} = -\frac{1}{\frac{P}{k_{\mathrm{d}, i}} + \frac{1}{k_{\mathrm{c}, i}}} \frac{A_{\mathrm{d}, i} (1-\phi_i)}{V_{\mathrm{d}, i}} \left( P C_{\mathrm{API}, \mathrm{d}, i} - C_{\mathrm{API}, \mathrm{c}} \right),
\end{equation}
where $C_{\mathrm{API}, \mathrm{d},i}$ is the API concentration in the $i$-th droplet, $C_{\mathrm{API}, \mathrm{c}}$ is the API concentration in the continuous phase, $P=C_{\mathrm{API}, \mathrm{c}}^*/C_{\mathrm{API}, \mathrm{d}}^*$ is the partition coefficient with $C_{\mathrm{API}, \mathrm{d}}^*$ and $C_{\mathrm{API}, \mathrm{c}}^*$ are the API concentration at the interface of the disperse and continuous phases, respectively, $k_{\mathrm{d}, i} = \mathrm{Sh} \mathcal{D}_{\mathrm{API}, \mathrm{d}} / d_{\mathrm{d}, i}$ is the mass transfer coefficient in the disperse phase, $k_{\mathrm{c}, i} = \mathrm{Sh} \mathcal{D}_{\mathrm{API}, \mathrm{c}} / d_{\mathrm{d}, i}$ is the mass transfer coefficient in the continuous phase, $\mathcal{D}_{\mathrm{API}, \mathrm{d}}$ and $\mathcal{D}_{\mathrm{API}, \mathrm{c}}$ are the API diffusivities in the disperse and continuous phases, respectively, $A_{\mathrm{d}, i}$, $V_{\mathrm{d}, i}$, $\phi_i$ are the surface area, volume, and coverage of the $i$-th droplet, respectively.

Accordingly, the governing equation for API concentration in the continuous phase can be written as:
\begin{equation}
    \frac{\mathrm{d} C_{\mathrm{API}, \mathrm{c}}}{\mathrm{d} t} = -\frac{1}{V_\mathrm{c}} \sum_{i=1}^{N_\mathrm{d}} \left( V_{\mathrm{d}, i} \frac{\mathrm{d} C_{\mathrm{API}, \mathrm{d},i}}{\mathrm{d} t} \right) = \sum_{i=1}^{N_\mathrm{d}} \left( \frac{1}{\frac{P}{k_{\mathrm{d}, i}} + \frac{1}{k_{\mathrm{c}, i}}} \frac{A_{\mathrm{d}, i} (1-\phi_i)}{V_{\mathrm{c}}} \left( P C_{\mathrm{API}, \mathrm{d}, i} - C_{\mathrm{API}, \mathrm{c}} \right) \right),
\end{equation}
where $V_\mathrm{c}$ is the volume of the continuous phase.

\subsection{Particle dissolution solved using population balance equation}

The dissolution process of Pickering particles can be described by population balance equation. Anisotropic Capsule-shaped particles require two internal coordinates to describe their length and diameter. Therefore, a bivariate particle size distribution (PSD) is adopted, $n(t, r_\mathrm{p}, l_\mathrm{c}$. The population balance equation for particle dissolution can be written as \cite{feng2023_Int.J.Multiph.Flow}:
\begin{equation}
\frac{\partial n}{\partial t} + \frac{\partial}{\partial r_\mathrm{p}} \left(\frac{\mathrm{d} r_\mathrm{p}}{\mathrm{d} t} n \right) + \frac{\partial}{\partial l_\mathrm{c}} \left(\frac{\mathrm{d} l_\mathrm{c}}{\mathrm{d} t} n \right) = 0
\label{eq: full pbe}
\end{equation}

As mentioned previously, $l_\mathrm{c}$ remains constant during particle dissolution, which gives $\mathrm{d} l_\mathrm{c}/\mathrm{d} t = 0$. Therefore, Eq.~\eqref{eq: full pbe} can be simplified as:
\begin{equation}
    \frac{\partial n}{\partial t} + \frac{\partial}{\partial r_\mathrm{p}} \left(\frac{\mathrm{d} r_\mathrm{p}}{\mathrm{d} t} n \right) = 0
\label{eq: pbe}
\end{equation}
where $\mathrm{d} r_\mathrm{p}/\mathrm{d} t$ can be calculated using Eq.~\eqref{eq: ddpdt}

\subsubsection{Sectional method for PBE}

The bivariate PBE (Eq.~\eqref{eq: pbe}) can be solved using sectional method (SM). The two internal coordinate spaces are discretized into two-dimensional grid. 

\begin{equation}
    N_{i,j} = \int_{r_{\mathrm{p}, i-1/2}}^{r_{\mathrm{p}, i+1/2}} \int_{l_{\mathrm{c}, j-1/2}}^{l_{\mathrm{c}, j+1/2}} n \,\mathrm{d} r_\mathrm{p} \,\mathrm{d} l_\mathrm{c}
\end{equation}

Then, the PSD can be written as:
\begin{equation}
    n\left(t, r_\mathrm{p}, l_\mathrm{c}\right) = \sum_{i=1}^{N_1} \sum_{j=1}^{N_2} N_{i,j}(t) \delta(r_\mathrm{p} - \mathcal{R}_{\mathrm{p}, i}) \delta(l_\mathrm{c} - \mathcal{L}_{\mathrm{c}, j})
\end{equation}
where $\mathcal{R}_{\mathrm{p}, i}$ and $\mathcal{L}_{\mathrm{c}, j}$ are the pivotal points in the intervals $[r_{\mathrm{p}, i-1/2}, r_{\mathrm{p}, i+1/2})$ and $[l_{\mathrm{c}, j-1/2}, l_{\mathrm{c}, j+1/2})$, respectively.

\subsubsection{Conditioned quadruture-based moment method for PBE}

The bivariate PBE (Eq.~\eqref{eq: pbe}) can be also solved using conditioned quadruture-based moment method (CQMOM). The moment equations can be obtained by integrating Eq.~\eqref{eq: pbe} over the two internal coordinates:
\begin{equation}
    \frac{\partial \mathcal{M}_{i,j}}{\partial t} + \int_0^{+\infty} \int_0^{+\infty} \frac{\partial}{\partial r_\mathrm{p}} \left(\frac{\mathrm{d} r_\mathrm{p}}{\mathrm{d} t} n \right) \,\mathrm{d} r_\mathrm{p} \, \mathrm{d} l_\mathrm{c} = 0
\end{equation}
where $\mathcal{M}_{i,j} = \int_0^{+\infty} \int_0^{+\infty} n \,\mathrm{d} r_\mathrm{p} \, \mathrm{d} l_\mathrm{c}$.



\section{Results and discussion}

\section{Conclusions}

The Elsevier cas-sc class is based on the
standard article class and supports almost all of the functionality of
that class. In addition, it features commands and options to format the
\begin{itemize} \item document style \item baselineskip \item front
matter \item keywords and MSC codes \item theorems, definitions and
proofs \item lables of enumerations \item citation style and labeling.
\end{itemize}

This class depends on the following packages
for its proper functioning:

\begin{enumerate}
\itemsep=0pt
\item {natbib.sty} for citation processing;
\item {geometry.sty} for margin settings;
\item {fleqn.clo} for left aligned equations;
\item {graphicx.sty} for graphics inclusion;
\item {hyperref.sty} optional packages if hyperlinking is
  required in the document;
\end{enumerate}  

All the above packages are part of any
standard \LaTeX{} installation.
Therefore, the users need not be
bothered about downloading any extra packages.

\section{Installation}

The package is available at author resources page at Elsevier
(\url{http://www.elsevier.com/locate/latex}).
The class may be moved or copied to a place, usually,\linebreak
\verb+$TEXMF/tex/latex/elsevier/+, %$%%%%%%%%%%%%%%%%%%%%%%%%%%%%
or a folder which will be read                   
by \LaTeX{} during document compilation.  The \TeX{} file
database needs updation after moving/copying class file.  Usually,
we use commands like \verb+mktexlsr+ or \verb+texhash+ depending
upon the distribution and operating system.

\section{Front matter}

The author names and affiliations could be formatted in two ways:
\begin{enumerate}[(1)]
\item Group the authors per affiliation.
\item Use footnotes to indicate the affiliations.
\end{enumerate}
See the front matter of this document for examples. 
You are recommended to conform your choice to the journal you 
are submitting to.

\section{Bibliography styles}

There are various bibliography styles available. You can select the
style of your choice in the preamble of this document. These styles are
Elsevier styles based on standard styles like Harvard and Vancouver.
Please use Bib\TeX\ to generate your bibliography and include DOIs
whenever available.

Here are two sample references: \cite{Fortunato2010}
\cite{Fortunato2010,NewmanGirvan2004}
\cite{Fortunato2010,Vehlowetal2013}

\section{Floats}
{Figures} may be included using the command,\linebreak 
\verb+\includegraphics+ in
combination with or without its several options to further control
graphic. \verb+\includegraphics+ is provided by {graphic[s,x].sty}
which is part of any standard \LaTeX{} distribution.
{graphicx.sty} is loaded by default. \LaTeX{} accepts figures in
the postscript format while pdf\LaTeX{} accepts {*.pdf},
{*.mps} (metapost), {*.jpg} and {*.png} formats. 
pdf\LaTeX{} does not accept graphic files in the postscript format. 

\begin{figure}
	\centering
		\includegraphics[scale=.75]{figs/Fig1.pdf}
	\caption{The evanescent light - $1S$ quadrupole coupling
	($g_{1,l}$) scaled to the bulk exciton-photon coupling
	($g_{1,2}$).}
	\label{FIG:1}
\end{figure}


The \verb+table+ environment is handy for marking up tabular
material. If users want to use {multirow.sty},
{array.sty}, etc., to fine control/enhance the tables, they
are welcome to load any package of their choice and
{cas-sc.cls} will work in combination with all loaded
packages.

\begin{table}[width=.9\linewidth,cols=4,pos=h]
\caption{This is a test caption. This is a test caption. This is a test
caption. This is a test caption.}\label{tbl1}
\begin{tabular*}{\tblwidth}{@{} LLLL@{} }
\toprule
Col 1 & Col 2 & Col 3 & Col4\\
\midrule
12345 & 12345 & 123 & 12345 \\
12345 & 12345 & 123 & 12345 \\
12345 & 12345 & 123 & 12345 \\
12345 & 12345 & 123 & 12345 \\
12345 & 12345 & 123 & 12345 \\
\bottomrule
\end{tabular*}
\end{table}

\section[Theorem and ...]{Theorem and theorem like environments}

{cas-sc.cls} provides a few shortcuts to format theorems and
theorem-like environments with ease. In all commands the options that
are used with the \verb+\newtheorem+ command will work exactly in the same
manner. {cas-sc.cls} provides three commands to format theorem or
theorem-like environments: 

\begin{verbatim}
 \newtheorem{theorem}{Theorem}
 \newtheorem{lemma}[theorem]{Lemma}
 \newdefinition{rmk}{Remark}
 \newproof{pf}{Proof}
 \newproof{pot}{Proof of Theorem \ref{thm2}}
\end{verbatim}


The \verb+\newtheorem+ command formats a
theorem in \LaTeX's default style with italicized font, bold font
for theorem heading and theorem number at the right hand side of the
theorem heading.  It also optionally accepts an argument which
will be printed as an extra heading in parentheses. 

\begin{verbatim}
  \begin{theorem} 
   For system (8), consensus can be achieved with 
   $\|T_{\omega z}$ ...
     \begin{eqnarray}\label{10}
     ....
     \end{eqnarray}
  \end{theorem}
\end{verbatim}  


\newtheorem{theorem}{Theorem}

\begin{theorem}
For system (8), consensus can be achieved with 
$\|T_{\omega z}$ ...
\begin{eqnarray}\label{10}
....
\end{eqnarray}
\end{theorem}

The \verb+\newdefinition+ command is the same in
all respects as its \verb+\newtheorem+ counterpart except that
the font shape is roman instead of italic.  Both
\verb+\newdefinition+ and \verb+\newtheorem+ commands
automatically define counters for the environments defined.

The \verb+\newproof+ command defines proof environments with
upright font shape.  No counters are defined. 


\section[Enumerated ...]{Enumerated and Itemized Lists}
{cas-sc.cls} provides an extended list processing macros
which makes the usage a bit more user friendly than the default
\LaTeX{} list macros.   With an optional argument to the
\verb+\begin{enumerate}+ command, you can change the list counter
type and its attributes.

\begin{verbatim}
 \begin{enumerate}[1.]
 \item The enumerate environment starts with an optional
   argument `1.', so that the item counter will be suffixed
   by a period.
 \item You can use `a)' for alphabetical counter and '(i)' 
  for roman counter.
  \begin{enumerate}[a)]
    \item Another level of list with alphabetical counter.
    \item One more item before we start another.
    \item One more item before we start another.
    \item One more item before we start another.
    \item One more item before we start another.
\end{verbatim}

Further, the enhanced list environment allows one to prefix a
string like `step' to all the item numbers.  

\begin{verbatim}
 \begin{enumerate}[Step 1.]
  \item This is the first step of the example list.
  \item Obviously this is the second step.
  \item The final step to wind up this example.
 \end{enumerate}
\end{verbatim}

\section{Cross-references}
In electronic publications, articles may be internally
hyperlinked. Hyperlinks are generated from proper
cross-references in the article.  For example, the words
\textcolor{black!80}{Fig.~1} will never be more than simple text,
whereas the proper cross-reference \verb+\ref{tiger}+ may be
turned into a hyperlink to the figure itself:
\textcolor{blue}{Fig.~1}.  In the same way,
the words \textcolor{blue}{Ref.~[1]} will fail to turn into a
hyperlink; the proper cross-reference is \verb+\cite{Knuth96}+.
Cross-referencing is possible in \LaTeX{} for sections,
subsections, formulae, figures, tables, and literature
references.

\section{Bibliography}

Two bibliographic style files (\verb+*.bst+) are provided ---
{model1-num-names.bst} and {model2-names.bst} --- the first one can be
used for the numbered scheme. This can also be used for the numbered
with new options of {natbib.sty}. The second one is for the author year
scheme. When  you use model2-names.bst, the citation commands will be
like \verb+\citep+,  \verb+\citet+, \verb+\citealt+ etc. However when
you use model1-num-names.bst, you may use only \verb+\cite+ command.

\verb+thebibliography+ environment.  Each reference is a\linebreak
\verb+\bibitem+ and each \verb+\bibitem+ is identified by a label,
by which it can be cited in the text:

\noindent In connection with cross-referencing and
possible future hyperlinking it is not a good idea to collect
more that one literature item in one \verb+\bibitem+.  The
so-called Harvard or author-year style of referencing is enabled
by the \LaTeX{} package {natbib}. With this package the
literature can be cited as follows:

\begin{enumerate}[\textbullet]
\item Parenthetical: \verb+\citep{WB96}+ produces (Wettig \& Brown, 1996).
\item Textual: \verb+\citet{ESG96}+ produces Elson et al. (1996).
\item An affix and part of a reference:\break
\verb+\citep[e.g.][Ch. 2]{Gea97}+ produces (e.g. Governato et
al., 1997, Ch. 2).
\end{enumerate}

In the numbered scheme of citation, \verb+\cite{<label>}+ is used,
since \verb+\citep+ or \verb+\citet+ has no relevance in the numbered
scheme.  {natbib} package is loaded by {cas-sc} with
\verb+numbers+ as default option.  You can change this to author-year
or harvard scheme by adding option \verb+authoryear+ in the class
loading command.  If you want to use more options of the {natbib}
package, you can do so with the \verb+\biboptions+ command.  For
details of various options of the {natbib} package, please take a
look at the {natbib} documentation, which is part of any standard
\LaTeX{} installation.

\appendix
% appendix setups
\renewcommand{\thesection}{\Alph{section}}
\numberwithin{equation}{section}
\numberwithin{figure}{section}
\numberwithin{table}{section}


\section{Dissolution rate of capsule-shaped particles}
\label{app: particle dissolution}
\subsection{Diameter change rate of capsule-shaped particles}

For one of the semi-spherical ends of a capsule-shaped particle, we have:
\begin{equation}
    \frac{\mathrm{d} m_{\mathrm{semi-sph}}}{\mathrm{d} t} = \frac{\mathrm{d}}{\mathrm{d} t} \left( \frac{1}{2} \left(\frac{1}{6} \pi d_\mathrm{p}^3 \rho_\mathrm{p} \right) \right) = \frac{1}{4} \pi \rho_\mathrm{p} d_\mathrm{p}^2 \frac{\mathrm{d} d_\mathrm{p} }{\mathrm{d} t}.
\end{equation}

Substitute the above equation and $A_\mathrm{semi-sph}=\left(\pi d_\mathrm{p}^2\right)/2$ into Eq.~\eqref{eq: particle mass transfer equation}, we have:
\begin{equation}
    \frac{1}{4} \pi \rho_\mathrm{p} d_\mathrm{p}^2 \frac{\mathrm{d} d_\mathrm{p} }{\mathrm{d} t} = -\frac{1}{2} \pi d_\mathrm{p}^2 k \left(C_\mathrm{sat} - C_{\mathrm{p}, \mathrm{c}} \right),
\end{equation}
which gives
\begin{equation}
    \frac{\mathrm{d} d_\mathrm{p} }{\mathrm{d} t} = -\frac{2k \left(C_\mathrm{sat} - C_{\mathrm{p}, \mathrm{c}} \right) }{\rho_\mathrm{p}}.
    \label{eq: dddt semi-sphere}
\end{equation}

Similarly, for the cylindrical section of the particle,
\begin{equation}
    \frac{\mathrm{d} m_{\mathrm{cyl}}}{\mathrm{d} t} = \frac{\mathrm{d}}{\mathrm{d} t} \left( \frac{1}{4} \pi d_\mathrm{p}^2 l_\mathrm{c} \rho_\mathrm{p} \right) = \frac{1}{2} \pi \rho_\mathrm{p} d_\mathrm{p} l_\mathrm{c} \frac{\mathrm{d} d_\mathrm{p} }{\mathrm{d} t},
\end{equation}
where $l_\mathrm{c}$ is the height of the cylindrical section, which is a constant because its top and bottom faces are covered by the semi-spherical ends. Since dissolution happens only at the lateral face of the cylindrical section, the dissolution surface area is $A_\mathrm{cyl}=\pi d_\mathrm{p} l_\mathrm{c}$. Substitute them into Eq.~\eqref{eq: particle mass transfer equation}, we have:
\begin{equation}
    \frac{1}{2} \pi \rho_\mathrm{p} d_\mathrm{p} l_\mathrm{c} \frac{\mathrm{d} d_\mathrm{p} }{\mathrm{d} t} = -k \pi d_\mathrm{p} l_\mathrm{c} \left(C_\mathrm{sat} - C_{\mathrm{p}, \mathrm{c}} \right),
\end{equation}
which gives
\begin{equation}
    \frac{\mathrm{d} d_\mathrm{p} }{\mathrm{d} t} = -\frac{2k \left(C_\mathrm{sat} - C_{\mathrm{p}, \mathrm{c}} \right) }{\rho_\mathrm{p}}.
    \label{eq: dddt cylinder}
\end{equation}

Eq.~\eqref{eq: dddt semi-sphere} and Eq.~\eqref{eq: dddt cylinder} are identical, indicating that the diameters of the semi-spherical ends and the cylindrical section of a capsule-shaped particle decrease at the same rate.

%\subsection{Available surface area for dissolution}

%For a capsule-shaped particle with a contact angle of $\theta$, the available dissolution area ratio $k_\mathrm{a}$ is defined as:
%\begin{eqnarray}
%    k_\mathrm{a}=\frac{A_\mathrm{c}}{A},
%\end{eqnarray}
%where $A_\mathrm{c}$ is the surface area of the particle in the continuous phase, and $A=\pi d_\mathrm{p}^2 + \pi d_\mathrm{p}l_\mathrm{c} $ is the surface area of the particle.

%\begin{equation}
%    A_\mathrm{c} = \pi d_\mathrm{p}^2 \frac{1 - \cos\theta}{2} + 2 \theta d_\mathrm{p} l_\mathrm{c},
%\end{equation}
%consequently,
%\begin{equation}
%    k_\mathrm{a} = \frac{\pi d_\mathrm{p}^2 \frac{1 - \cos\theta}{2} + \theta d_\mathrm{p} l_\mathrm{c}}{\pi d_\mathrm{p}^2 + \pi d_\mathrm{p}l_\mathrm{c}}
%\end{equation}

\printcredits

%% Loading bibliography style file
\bibliographystyle{model1-num-names}
%\bibliographystyle{cas-model2-names}

% Loading bibliography database
\bibliography{api_release}


%\vskip3pt

\end{document}
