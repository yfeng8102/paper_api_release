%% 
%% Copyright 2019-2020 Elsevier Ltd
%% 
%% This file is part of the 'CAS Bundle'.
%% --------------------------------------
%% 
%% It may be distributed under the conditions of the LaTeX Project Public
%% License, either version 1.2 of this license or (at your option) any
%% later version.  The latest version of this license is in
%%    http://www.latex-project.org/lppl.txt
%% and version 1.2 or later is part of all distributions of LaTeX
%% version 1999/12/01 or later.
%% 
%% The list of all files belonging to the 'CAS Bundle' is
%% given in the file `manifest.txt'.
%% 
%% Template article for cas-sc documentclass for 
%% double column output.

%\documentclass[a4paper,fleqn,longmktitle]{cas-sc}
\documentclass[a4paper,fleqn]{cas-sc}

\usepackage[numbers]{natbib}
%\usepackage[authoryear]{natbib}
%\usepackage[authoryear,longnamesfirst]{natbib}

% my package
\usepackage{lineno}
\usepackage{setspace}

%%%Author definitions
%\def\tsc#1{\csdef{#1}{\textsc{\lowercase{#1}}\xspace}}
%\tsc{WGM}
%\tsc{QE}
%\tsc{EP}
%\tsc{PMS}
%\tsc{BEC}
%\tsc{DE}
%%%

% Uncomment and use as if needed
%\newtheorem{theorem}{Theorem}
%\newtheorem{lemma}[theorem]{Lemma}
%\newdefinition{rmk}{Remark}
%\newproof{pf}{Proof}
%\newproof{pot}{Proof of Theorem \ref{thm}}

\begin{document}

% my command
\linenumbers

\let\WriteBookmarks\relax
\def\floatpagepagefraction{1}
\def\textpagefraction{.001}

% Short title
\shorttitle{API release of Pickering emulsions}

% Short author
\shortauthors{Y. Feng et~al.}

% Main title of the paper
\title [mode = title]{Monte Carlo simulation of Pickering particle dissolution and API release in Pickering Emulsions}
% Title footnote mark
% eg: \tnotemark[1]
%\tnotemark[1,2]

% Title footnote 1.
% eg: \tnotetext[1]{Title footnote text}
% \tnotetext[<tnote number>]{<tnote text>} 
%\tnotetext[1]{This document is the results of the research project funded by the National Science Foundation.}

%\tnotetext[2]{The second title footnote which is a longer text matter to fill through the whole text width and overflow into another line in the footnotes area of the first page.}


% First author
%
% Options: Use if required
% eg: \author[1,3]{Author Name}[type=editor,
%       style=chinese,
%       auid=000,
%       bioid=1,
%       prefix=Sir,
%       orcid=0000-0000-0000-0000,
%       facebook=<facebook id>,
%       twitter=<twitter id>,
%       linkedin=<linkedin id>,
%       gplus=<gplus id>]

% First author
\author[]{Yi Feng}[orcid=0000-0001-9359-567X]
\ead{yi.feng@polito.it}
\credit{Conceptualization of this study, Methodology, Software}

% Second author
\author[]{Antonio Buffo}[orcid=0000-0002-4152-0593]
\ead{antonio.buffo@polito.it}
\credit{Data curation, Writing - Original draft preparation}

\author[]{Elena Simone}[orcid=0000-0003-4000-2222]
\ead{elena.simone@polito.it}
\credit{Data curation, Writing - Original draft preparation}
\cormark[1]

% Address/affiliation
\affiliation[]{organization={Department of Applied Science and Technology, Politecnico di Torino},
    addressline={Corso Duca degli Abruzzi, 24},
    city={Torino},
    % citysep={}, % Uncomment if no comma needed between city and postcode
    postcode={10129},
    % state={},
    country={Italy}}

% Corresponding author text
\cortext[cor1]{Corresponding author}

% Here goes the abstract
\begin{abstract}
    This template helps you to create a properly formatted \LaTeX\ manuscript.

    \noindent\texttt{\textbackslash begin{abstract}} \dots
    \texttt{\textbackslash end{abstract}} and
    \verb+\begin{keyword}+ \verb+...+ \verb+\end{keyword}+
    which
    contain the abstract and keywords respectively.

    \noindent Each keyword shall be separated by a \verb+\sep+ command.
\end{abstract}

% Use if graphical abstract is present
% \begin{graphicalabstract}
% \includegraphics{figs/grabs.pdf}
% \end{graphicalabstract}

% Research highlights
\begin{highlights}
    \item Random sequential adsorption gives coverages of Pickering emulsions
    \item API release from a population of Pickering emulsions is studied
    \item Pickering particle dissolution induced coverage changes are included
\end{highlights}

% Keywords
% Each keyword is seperated by \sep
\begin{keywords}
    Pickering emulsions \sep API release \sep particle dissolution \sep Monte Carlo method
\end{keywords}


\maketitle

\doublespacing

\section{Introduction}



\section{Methodology}



\subsection{Geometric modeling of Pickering emulsion}

The coverage of Pickering emulsion droplets plays an important role in API release, because it affects the available surface area for API release. The coverage is determined by the adsorption of Pickering particles on the droplet surfaces. Unfortunately, measurement of coverage is challenging due to its multiscale multiphase physics. Therefore, numerical simulation is adopted in this work to obtain the coverage of Pickering emulsion droplets. Depending on the packing pattern of Pickering particles, different numerical methods can be used to model the adsorption process of Pickering particles.

Based on our experiments, this study focuses on the micron-sized elongated curcumin particles, which can be approximated as capsule-shaped particles consisting of two semi-spherical ends and a cylindrical section. A loose random packing pattern of these particles is observed on the emulsion droplet surfaces, which can be modeled using random sequential adsorption (RSA) method. In RSA, particles sequentially attempt to adsorb at random locations on the droplet surfaces. They are adsorbed only if they do not overlap with any previously adsorbed particles, and once adsorbed, they remain fixed in place. The RSA simulation can provide a detailed records of the adsorbed Pickering particles, including their size distribution and the coverage they provide, which are valuable for modeling the API release from Pickering emulsions.

The RSA approach is implemented using Monte Carlo method. Meanwhile, the probability of a successful adsorption reduces significantly as coverage increases, which makes the RSA simulations computationally expensive. The jamming state, which is the maximum coverage that can be achieved by RSA, is rarely reached in the simulations, especially for anisotropic particles. Therefore, this study adopts a extrapolation method to obtain the jamming coverage, $\phi_\mathrm{j}$. The RSA simulation is performed to reach a near-jamming state. The obtained coverage evolution curve is fitted to extrapolate the jamming coverage. Moreover, the size distribution of the adsorbed particles at the end of simulation is used to represent the size distribution of the adsorbed particles at the jamming state. These jamming coverage and particle size distribution are used to model the API release from Pickering emulsions.

Details about the RSA simulation and extrapolation method for $\phi_\mathrm{j}$ can be found in \cite{feng2025_Phys.Rev.E}.


\subsection{Dissolution of Pickering particles}

\subsubsection{Particle-resolved dissolution model}

The dissolution of a particle can be described by the following mass transfer equation \cite{bird2002_}:
\begin{equation}
    \frac{\mathrm{d} m_\mathrm{p}}{\mathrm{d} t} = - k_\mathrm{p} A_\mathrm{p} \left(C_\mathrm{sat} - C_{\mathrm{p}, \mathrm{c}} \right),
    \label{eq: particle mass transfer equation}
\end{equation}
where $m_\mathrm{p}$ is the particle mass, $k_\mathrm{p}$ is the particle mass transfer coefficient, $A_\mathrm{p}$ is the particle surface area, $C_{\mathrm{sat}}$ is the saturation concentration, and $C_{\mathrm{p}, \mathrm{c}}$ is the concentration in the continuous (bulk) phase.

For a capsule-shaped particle consisting of two semi-spherical ends and a cylindrical section, it is natural to assume that the particle surface shrinks along its normal direction during dissolution. Therefore, the height of the cylindrical section, $l_\mathrm{c}$, remains constant during dissolution because the top and bottom faces of the cylindrical section are covered by the semi-spherical ends. It can be proved that the particle surface, of both the semi-spherical and cylindrical parts, shrinks at the same rate as (see Appendix \ref{app: particle dissolution}):
\begin{equation}
    \frac{\mathrm{d} d_\mathrm{p} }{\mathrm{d} t} = -\frac{2 k_\mathrm{p} \left(C_\mathrm{sat} - C_{\mathrm{p}, \mathrm{c}} \right) }{\rho_\mathrm{p}},
    \label{eq: particle diameter change rate}
\end{equation}
where $\rho_\mathrm{p}$ is the particle density. The dissolution process can be governed by the reaction mechanism or the diffusion mechanism \cite{mullin2001_}. For reaction-controlled dissolution, it is assumed to be a first-order process, and a constant particle mass transfer coefficient $k_\mathrm{p}$ will be used. For a diffusion-controlled dissolution, the particle mass transfer coefficient can be written as:
\begin{equation}
    k_\mathrm{p} = \frac{\mathrm{Sh} \mathcal{D}_{\mathrm{p}, \mathrm{c}}}{d_{32, \mathrm{p}}},
\end{equation}
where $\mathrm{Sh}=2$ is the Sherwood number, $\mathcal{D}_{\mathrm{p}, \mathrm{c}}$ is the particle diffusivity in the continuous phase, and $d_{32, \mathrm{p}}=6V_\mathrm{p}/A_\mathrm{p}$ is the Sauter mean diameter of the particle with $V_\mathrm{p}=\pi d_\mathrm{p}^3/6 + \pi d_\mathrm{p}^2l_\mathrm{c}/4$ be the particle volume and $A_\mathrm{p}=\pi d_\mathrm{p}^2 + \pi d_\mathrm{p}l_\mathrm{c}$ be the particle surface area. The reaction-controlled dissolution process is size-independent, while the diffusion-controlled dissolution process is size-dependent. In experiments, the reacting $k_\mathrm{p}$ is generally easier to measure by monitoring the particle concentration in the continuous phase. The diffusing $k_\mathrm{p}$ is relatively difficult to measure because it requires accurate measurement of particle diffusivity and size distribution in addition to particle concentration. Based on this consideration, the reaction mechanism will be adopted in most cases, whereas the diffusion mechanism will be used only for comparison with the reaction mechanism.

Furthermore, Pickering particles are adsorbed at the Pickering emulsion droplet surfaces. Only a porportion of the particle surface is exposed into the continuous phase, while the rest is inside the dispersed droplet. In this work, particles are assumed to be only dissoluble in the continuous phase. The proportion exposed to the continuous phase affects the dissolution rate. This proportion depends on the particle contact angle. In this work, the proportion of the available surface area for dissolution is calculated using the following equation
\begin{equation}
    k_\mathrm{a} = \frac{\pi d_\mathrm{p}^2 \frac{1 - \cos\theta}{2} + \theta d_\mathrm{p} l_\mathrm{c}}{\pi d_\mathrm{p}^2 + \pi d_\mathrm{p}l_\mathrm{c}}
    \label{eq: ka}
\end{equation}
where $\theta$ is the contact angle of particle. Eq.~\eqref{eq: ka} depends only on particle sizes and contact angle, which implies a planar interface.

For the $i$-th droplet, the diameter and mass change rate of its $j$-th adsorbed particle are written as:
\begin{equation}
    \frac{\mathrm{d} d_{\mathrm{p},j}^{(i)}}{\mathrm{d} t}
    = -\frac{2 k_{\mathrm{p},j}^{(i)} k_{\mathrm{a},j}^{(i)} \left(C_\mathrm{sat} - C_{\mathrm{p},\mathrm{c}} \right)}{\rho_\mathrm{p}},
    \label{eq: ddpdt}
\end{equation}
\begin{equation}
    \frac{\mathrm{d} m_{\mathrm{p},j}^{(i)}}{\mathrm{d} t}
    = \frac{\pi}{2}\rho_\mathrm{p} \left[\left(d_{\mathrm{p},j}^{(i)}\right)^2 + d_{\mathrm{p},j}^{(i)} l_{\mathrm{c},j}^{(i)}\right]
    \frac{\mathrm{d} d_{\mathrm{p},j}^{(i)}}{\mathrm{d} t}.
\end{equation}

Particle dissolution affects API release through coverage. For the $i$-th droplet, the coverage is
\begin{equation}
    \phi^{(i)} = \sum_{j=1}^{N_{\mathrm{p}}^{(i)}} \varphi\!\left(r_{\mathrm{p},j}^{(i)}, l_{\mathrm{c},j}^{(i)}, \theta, r_{\mathrm{d}}^{(i)}\right),
\end{equation}
where $\varphi$ is the coverage contribution of one adsorbed particle as a function of particle size, contact angle, and droplet size, and $N_{\mathrm{p}}^{(i)}$ is the number of particles on droplet $i$. The detailed numerical evaluation of $\varphi$ is provided in \cite{feng2025_Phys.Rev.E}.


\subsection{Particle dissolution solved using population balance equation}

The dissolution process of Pickering particles can be described by population balance equation. Anisotropic Capsule-shaped particles require two internal coordinates to describe their length and diameter. Therefore, a bivariate particle size distribution (PSD) is adopted, $n(t, r_\mathrm{p}, l_\mathrm{c}$. The population balance equation for particle dissolution can be written as \cite{feng2023_Int.J.Multiph.Flow}:
\begin{equation}
    \frac{\partial n}{\partial t} + \frac{\partial}{\partial r_\mathrm{p}} \left(\frac{\mathrm{d} r_\mathrm{p}}{\mathrm{d} t} n \right) + \frac{\partial}{\partial l_\mathrm{c}} \left(\frac{\mathrm{d} l_\mathrm{c}}{\mathrm{d} t} n \right) = 0
    \label{eq: full pbe}
\end{equation}

As mentioned previously, $l_\mathrm{c}$ remains constant during particle dissolution, which gives $\mathrm{d} l_\mathrm{c}/\mathrm{d} t = 0$. Therefore, Eq.~\eqref{eq: full pbe} can be simplified as:
\begin{equation}
    \frac{\partial n}{\partial t} + \frac{\partial}{\partial r_\mathrm{p}} \left(\frac{\mathrm{d} r_\mathrm{p}}{\mathrm{d} t} n \right) = 0
    \label{eq: pbe}
\end{equation}
where $\mathrm{d} r_\mathrm{p}/\mathrm{d} t$ can be calculated using Eq.~\eqref{eq: ddpdt}.


\subsubsection{Sectional method for PBE}

The bivariate PBE (Eq.~\eqref{eq: pbe}) can be solved using sectional method (SM). The two internal coordinate spaces are discretized into two-dimensional grid.

\begin{equation}
    N_{i,j} = \int_{r_{\mathrm{p}, i-1/2}}^{r_{\mathrm{p}, i+1/2}} \int_{l_{\mathrm{c}, j-1/2}}^{l_{\mathrm{c}, j+1/2}} n \,\mathrm{d} r_\mathrm{p} \,\mathrm{d} l_\mathrm{c}
\end{equation}

Then, the PSD can be written as:
\begin{equation}
    n\left(t, r_\mathrm{p}, l_\mathrm{c}\right) = \sum_{i=1}^{N_1} \sum_{j=1}^{N_2} N_{i,j}(t) \delta(r_\mathrm{p} - \mathcal{R}_{\mathrm{p}, i}) \delta(l_\mathrm{c} - \mathcal{L}_{\mathrm{c}, j})
\end{equation}
where $\mathcal{R}_{\mathrm{p}, i}$ and $\mathcal{L}_{\mathrm{c}, j}$ are the pivotal points in the intervals $[r_{\mathrm{p}, i-1/2}, r_{\mathrm{p}, i+1/2})$ and $[l_{\mathrm{c}, j-1/2}, l_{\mathrm{c}, j+1/2})$, respectively.

\subsubsection{Conditioned quadruture-based moment method for PBE}

The bivariate PBE (Eq.~\eqref{eq: pbe}) can be also solved using conditioned quadruture-based moment method (CQMOM). The moment equations can be obtained by integrating Eq.~\eqref{eq: pbe} over the two internal coordinates:
\begin{equation}
    \frac{\partial \mathcal{M}_{i,j}}{\partial t} + \int_0^{+\infty} \int_0^{+\infty} \frac{\partial}{\partial r_\mathrm{p}} \left(\frac{\mathrm{d} r_\mathrm{p}}{\mathrm{d} t} n \right) \,\mathrm{d} r_\mathrm{p} \, \mathrm{d} l_\mathrm{c} = 0
\end{equation}
where $\mathcal{M}_{i,j} = \int_0^{+\infty} \int_0^{+\infty} n \,\mathrm{d} r_\mathrm{p} \, \mathrm{d} l_\mathrm{c}$.

\subsubsection{Extension to a population of droplets}

For a population of $N_\mathrm{d}$ droplets, the continuous-phase particle concentration follows mass conservation:
\begin{equation}
    \frac{\mathrm{d} C_{\mathrm{p},\mathrm{c}}}{\mathrm{d} t}
    = -\frac{1}{V_\mathrm{c}} \sum_{i=1}^{N_\mathrm{d}} \sum_{j=1}^{N_{\mathrm{p}}^{(i)}} \frac{\mathrm{d} m_{\mathrm{p},j}^{(i)}}{\mathrm{d} t},
\end{equation}
where $V_\mathrm{c}$ is the continuous-phase volume.

\subsection{API release from Pickering emulsion droplets}

The release of API from the $i$-th dispersed droplet to the continuous phase can be described by
\begin{equation}
    \frac{\mathrm{d} C_{\mathrm{API},\mathrm{d}}^{(i)}}{\mathrm{d} t}
    = -\frac{1}{\frac{P}{k_{\mathrm{d}}^{(i)}} + \frac{1}{k_{\mathrm{c}}^{(i)}}}
    \frac{A_{\mathrm{d}}^{(i)} \left(1-\phi^{(i)}\right)}{V_{\mathrm{d}}^{(i)}}
    \left(P C_{\mathrm{API},\mathrm{d}}^{(i)} - C_{\mathrm{API},\mathrm{c}}\right),
\end{equation}
where $C_{\mathrm{API},\mathrm{d}}^{(i)}$ is the API concentration in the $i$-th droplet, $C_{\mathrm{API},\mathrm{c}}$ is the API concentration in the continuous phase, $P=C_{\mathrm{API},\mathrm{c}}^*/C_{\mathrm{API},\mathrm{d}}^*$ is the partition coefficient, and $A_{\mathrm{d}}^{(i)}$, $V_{\mathrm{d}}^{(i)}$, and $\phi^{(i)}$ are the surface area, volume, and coverage of the $i$-th droplet, respectively.

The API mass transfer coefficient in the dispersed phase and continuous phase are $k_{\mathrm{d}}^{(i)}$ and $k_{\mathrm{c}}^{(i)}$, respectively, which can be calculated as:
\begin{equation}
    k_{\mathrm{d}}^{(i)} = \frac{\mathrm{Sh}\mathcal{D}_{\mathrm{API},\mathrm{d}}}{d_{\mathrm{d}}^{(i)}},
    \qquad
    k_{\mathrm{c}}^{(i)} = \frac{\mathrm{Sh}\mathcal{D}_{\mathrm{API},\mathrm{c}}}{d_{\mathrm{d}}^{(i)}},
\end{equation}
where $\mathcal{D}_{\mathrm{API},\mathrm{d}}$ and $\mathcal{D}_{\mathrm{API},\mathrm{c}}$ are the API diffusivity in the dispersed phase and continuous phase, respectively, and $d_{\mathrm{d}}^{(i)}$ is the diameter of the $i$-th droplet. The Sherwood number is assumed to be 2 for all droplets, which corresponds to a diffusion-controlled mass transfer process with a stagnant layer around the droplet.

According to API mass conservation, the governing equation for API concentration in the continuous phase can be written as:
\begin{equation}
    \frac{\mathrm{d} C_{\mathrm{API},\mathrm{c}}}{\mathrm{d} t}
    = -\frac{1}{V_\mathrm{c}} \sum_{i=1}^{N_\mathrm{d}} V_{\mathrm{d}}^{(i)} \frac{\mathrm{d} C_{\mathrm{API},\mathrm{d}}^{(i)}}{\mathrm{d} t}
    = \sum_{i=1}^{N_\mathrm{d}}
    \frac{1}{\frac{P}{k_{\mathrm{d}}^{(i)}} + \frac{1}{k_{\mathrm{c}}^{(i)}}}
    \frac{A_{\mathrm{d}}^{(i)}\left(1-\phi^{(i)}\right)}{V_\mathrm{c}}
    \left(P C_{\mathrm{API},\mathrm{d}}^{(i)} - C_{\mathrm{API},\mathrm{c}}\right),
\end{equation}
where $V_\mathrm{c}$ is the volume of the continuous phase.






\section{Results and discussion}

\section{Conclusions}


\appendix
% appendix setups
\renewcommand{\thesection}{\Alph{section}}
\numberwithin{equation}{section}
\numberwithin{figure}{section}
\numberwithin{table}{section}


\section{Dissolution rate of capsule-shaped particles}
\label{app: particle dissolution}
\subsection{Diameter change rate of capsule-shaped particles}

For one of the semi-spherical ends of a capsule-shaped particle, we have:
\begin{equation}
    \frac{\mathrm{d} m_{\mathrm{semi-sph}}}{\mathrm{d} t} = \frac{\mathrm{d}}{\mathrm{d} t} \left( \frac{1}{2} \left(\frac{1}{6} \pi d_\mathrm{p}^3 \rho_\mathrm{p} \right) \right) = \frac{1}{4} \pi \rho_\mathrm{p} d_\mathrm{p}^2 \frac{\mathrm{d} d_\mathrm{p} }{\mathrm{d} t}.
\end{equation}

Substitute the above equation and $A_\mathrm{semi-sph}=\left(\pi d_\mathrm{p}^2\right)/2$ into Eq.~\eqref{eq: particle mass transfer equation}, we have:
\begin{equation}
    \frac{1}{4} \pi \rho_\mathrm{p} d_\mathrm{p}^2 \frac{\mathrm{d} d_\mathrm{p} }{\mathrm{d} t} = -\frac{1}{2} \pi d_\mathrm{p}^2 k \left(C_\mathrm{sat} - C_{\mathrm{p}, \mathrm{c}} \right),
\end{equation}
which gives
\begin{equation}
    \frac{\mathrm{d} d_\mathrm{p} }{\mathrm{d} t} = -\frac{2k \left(C_\mathrm{sat} - C_{\mathrm{p}, \mathrm{c}} \right) }{\rho_\mathrm{p}}.
    \label{eq: dddt semi-sphere}
\end{equation}

Similarly, for the cylindrical section of the particle,
\begin{equation}
    \frac{\mathrm{d} m_{\mathrm{cyl}}}{\mathrm{d} t} = \frac{\mathrm{d}}{\mathrm{d} t} \left( \frac{1}{4} \pi d_\mathrm{p}^2 l_\mathrm{c} \rho_\mathrm{p} \right) = \frac{1}{2} \pi \rho_\mathrm{p} d_\mathrm{p} l_\mathrm{c} \frac{\mathrm{d} d_\mathrm{p} }{\mathrm{d} t},
\end{equation}
where $l_\mathrm{c}$ is the height of the cylindrical section, which is a constant because its top and bottom faces are covered by the semi-spherical ends. Since dissolution happens only at the lateral face of the cylindrical section, the dissolution surface area is $A_\mathrm{cyl}=\pi d_\mathrm{p} l_\mathrm{c}$. Substitute them into Eq.~\eqref{eq: particle mass transfer equation}, we have:
\begin{equation}
    \frac{1}{2} \pi \rho_\mathrm{p} d_\mathrm{p} l_\mathrm{c} \frac{\mathrm{d} d_\mathrm{p} }{\mathrm{d} t} = -k \pi d_\mathrm{p} l_\mathrm{c} \left(C_\mathrm{sat} - C_{\mathrm{p}, \mathrm{c}} \right),
\end{equation}
which gives
\begin{equation}
    \frac{\mathrm{d} d_\mathrm{p} }{\mathrm{d} t} = -\frac{2k \left(C_\mathrm{sat} - C_{\mathrm{p}, \mathrm{c}} \right) }{\rho_\mathrm{p}}.
    \label{eq: dddt cylinder}
\end{equation}

Eq.~\eqref{eq: dddt semi-sphere} and Eq.~\eqref{eq: dddt cylinder} are identical, indicating that the diameters of the semi-spherical ends and the cylindrical section of a capsule-shaped particle decrease at the same rate.

%\subsection{Available surface area for dissolution}

%For a capsule-shaped particle with a contact angle of $\theta$, the available dissolution area ratio $k_\mathrm{a}$ is defined as:
%\begin{eqnarray}
%    k_\mathrm{a}=\frac{A_\mathrm{c}}{A},
%\end{eqnarray}
%where $A_\mathrm{c}$ is the surface area of the particle in the continuous phase, and $A=\pi d_\mathrm{p}^2 + \pi d_\mathrm{p}l_\mathrm{c} $ is the surface area of the particle.

%\begin{equation}
%    A_\mathrm{c} = \pi d_\mathrm{p}^2 \frac{1 - \cos\theta}{2} + 2 \theta d_\mathrm{p} l_\mathrm{c},
%\end{equation}
%consequently,
%\begin{equation}
%    k_\mathrm{a} = \frac{\pi d_\mathrm{p}^2 \frac{1 - \cos\theta}{2} + \theta d_\mathrm{p} l_\mathrm{c}}{\pi d_\mathrm{p}^2 + \pi d_\mathrm{p}l_\mathrm{c}}
%\end{equation}

\printcredits

%% Loading bibliography style file
\bibliographystyle{model1-num-names}
%\bibliographystyle{cas-model2-names}

% Loading bibliography database
\bibliography{api_release}


%\vskip3pt

\end{document}
